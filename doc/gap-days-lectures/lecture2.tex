\newcommand{\QPAIntroPartNumber}{2}
\documentclass[usenames,dvipsnames]{beamer}

%\usetheme{ntnu}
\usetheme{Warsaw}

\usepackage{times}
\usepackage[T1]{fontenc}
\usepackage[all]{xy}
\usepackage{textpos}
\usepackage{tikz}

\newcommand{\defn}[1]{\textit{#1}}
\newcommand{\Q}{\mathbb{Q}}
\newcommand{\equivalence}{\simeq}
\newcommand{\VV}[2]{\begin{pmatrix} #1 & #2 \end{pmatrix}}
\newcommand{\vv}[2]{\left( \begin{smallmatrix} #1 & #2 \end{smallmatrix} \right)}
\newcommand{\into}{\hookrightarrow}
\newcommand{\iso}{\cong}
\newcommand{\dsum}{\oplus}
\DeclareMathOperator{\fmod}{mod}
\DeclareMathOperator{\Rep}{Rep}
\DeclareMathOperator{\Hom}{Hom}
\DeclareMathOperator{\coker}{coker}
\DeclareMathOperator{\im}{im}
\DeclareMathOperator{\rad}{rad}
\DeclareMathOperator{\soc}{soc}
\DeclareMathOperator{\Top}{top}

\title[Introduction to QPA, part \QPAIntroPartNumber]
      {Introduction to QPA}
\subtitle{Part \QPAIntroPartNumber}

\author{\O{}ystein~Skarts\ae{}terhagen \and \O{}yvind~Solberg}
\institute{
Department of Mathematical Sciences\\
Norwegian University of Science and Technology}

\date{Third GAP Days}

% If you have a file called "university-logo-filename.xxx", where xxx
% is a graphic format that can be processed by latex or pdflatex,
% resp., then you can add a logo as follows:

% \pgfdeclareimage[height=0.5cm]{university-logo}{university-logo-filename}
% \logo{\pgfuseimage{university-logo}}



% Delete this, if you do not want the table of contents to pop up at
% the beginning of each subsection:
% \AtBeginSubsection[]
% {
%   \begin{frame}<beamer>{Outline}
%     \tableofcontents[currentsection,currentsubsection]
%   \end{frame}
% }


\begin{document}

\begin{frame}
  \titlepage
\end{frame}

\begin{frame}{Outline}
  \tableofcontents
\end{frame}

\section{Basic functions}

\subsection{Algebras}

\begin{frame}
TODO!
\end{frame}

\subsection{Modules}

\begin{frame}[fragile]{Recall: Modules (representations) in QPA}
\[
Q \colon
\xymatrix{1 \ar[r]^a & 2 \ar[r]^b & 3}
\]
\[
M \colon
\xymatrix{
k   \ar[r]^{\left( \begin{smallmatrix} 2 & 0 \end{smallmatrix} \right)} &
k^2 \ar[r]^{\left( \begin{smallmatrix} 4 \\ -1 \end{smallmatrix} \right)} &
k
}
\]
\begin{verbatim}
gap> Q := Quiver(3, [[1,2,"a"],[2,3,"b"]]);;
gap> kQ := PathAlgebra(Rationals, Q);;
gap> M := RightModuleOverPathAlgebra
          (kQ, [1,2,1],
           [["a", [[2,0]]], ["b", [[4],[-1]]]]);
<[ 1, 2, 1 ]>
\end{verbatim}
\end{frame}

\begin{frame}{Module attributes}
\begin{overprint}
\onslide<1>
\[
M \colon
\xymatrix{
{\textcolor{OliveGreen}{k}^{\textcolor{blue}{1}}}
\ar[r]^{\textcolor{BrickRed}
        {\left( \begin{smallmatrix} 2 & 0 \end{smallmatrix}
         \right)}} &
{\textcolor{OliveGreen}{k}^{\textcolor{blue}{2}}}
\ar[r]^{\textcolor{BrickRed}
        {\left( \begin{smallmatrix} 4 \\ -1 \end{smallmatrix}
         \right)}} &
{\textcolor{OliveGreen}{k}^{\textcolor{blue}{1}}}
}
\]
\begin{itemize}
\item \texttt{RightActingAlgebra}: $kQ$
\item \texttt{LeftActingDomain}: \textcolor{OliveGreen}{$k$}
\item \texttt{DimensionVector}: 
      $(\textcolor{blue}{1}, \textcolor{blue}{2},
        \textcolor{blue}{1})$
\item \texttt{MatricesOfPathAlgebraModule}:
      $\left(
        \textcolor{BrickRed}{
          \begin{pmatrix} 2 & 0 \end{pmatrix}},
        \textcolor{BrickRed}{
          \begin{pmatrix} 4 \\ -1 \end{pmatrix}}
      \right)$
\item \texttt{Dimension}:
      $4 = \textcolor{blue}{1}
           + \textcolor{blue}{2}
           + \textcolor{blue}{1}$
\end{itemize}
\onslide<2>
\[
M \colon
\xymatrix{
k^1
\ar[r]^{\left( \begin{smallmatrix} 2 & 0 \end{smallmatrix}
        \right)} &
k^2
\ar[r]^{\left( \begin{smallmatrix} 4 \\ -1 \end{smallmatrix}
        \right)} &
k^2
}
\]
\begin{itemize}
\item \texttt{Basis}:
\\ \vspace{-3em}
\begin{gather*}
1 \to (0,0) \to 0 \\
0 \to (1,0) \to 0 \\
0 \to (0,1) \to 0 \\
0 \to (0,0) \to 1
\end{gather*}
\item \texttt{MinimalGeneratingSetOfModule}:
\begin{gather*}
1 \to (0,0) \to 0 \\
0 \to (0,0) \to 1
\end{gather*}
\end{itemize}
\end{overprint}
\end{frame}

\begin{frame}[fragile]{Submodules}{}
\begin{overprint}
\onslide<1>
\relax{\huge
\[
N \stackrel{i}{\into} M
\]}
\begin{itemize}
\item Categorical view of submodules
\item A submodule is given by an inclusion homomorphism
\item A submodule is not a subset
\end{itemize}
\onslide<2>
\relax{\huge
\[
\textcolor{BrickRed}{N} \stackrel{\textcolor{blue}{i}}{\into} M
\]}
\begin{itemize}
\item Categorical view of submodules
\item A submodule is given by an inclusion homomorphism
\item A submodule is not a subset
\item \texttt{SubRepresentation}: \textcolor{BrickRed}{$N$}
\item \texttt{SubRepresentationInclusion}: \textcolor{blue}{$i$}
\end{itemize}
% \onslide<3>
% \relax{\huge
% \[
% N \stackrel{i}{\into} M
% \]}
% \begin{verbatim}
% gap> kQ := PathAlgebra(Rationals, Quiver(2, [[1,2,"a"],[1,2,"b"]]));;
% <Rationals[<quiver with 2 vertices and 2 arrows>]>
% gap> M := RightModuleOverPathAlgebra(kQ, [2,2], [["a", [[1,0],[1,1]]]]);
% <[ 2, 2 ]>
% gap> SubRepresentation(M, [Basis(M)[1]-Basis(M)[2]]);
% <[ 1, 1 ]>
% gap> SubRepresentationInclusion(M, [Basis(M)[1]-Basis(M)[2]]);
% <<[ 1, 1 ]> ---> <[ 2, 2 ]>>
% \end{verbatim}
%% [show example from QPA intro]
\end{overprint}
\end{frame}


\subsection{Homomorphisms}

\begin{frame}
TODO!
\end{frame}


\section{Chain complexes}

\begin{frame}{Chain complexes}
\end{frame}

\end{document}
