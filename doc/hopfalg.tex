%%%%%%%%%%%%%%%%%%%%%%%%%%%%%%%%%%%%%%%%%%%%%%%%%%%%%%%%%%%%%%%%%%%%%%%%%%%%%%
% QPA Project User Documentation
% DESCRIPTION: hopfalg.tex:  The documentation of the Hopf implementation of  
%              Hopf algebras and algebra antihomomorphisms.
%
% Copyright, 1998 Virginia Polytechnic Institute and State University.
% Copyright, 1998 Virginia Tech Hopf Project. All rights reserved.
%
% This file may be distributed in accordance with the stipulations existing
% in the LICENSE file accompanying this software.
%
% $Id: hopfalg.tex,v 1.1 2010/05/07 13:16:24 sunnyquiver Exp $
%%%%%%%%%%%%%%%%%%%%%%%%%%%%%%%%%%%%%%%%%%%%%%%%%%%%%%%%%%%%%%%%%%%%%%%%%%%%%%

\Chapter{Hopf Algebras}

A Hopf algebra is a bialgebra together with a special
map called the antipode.  The bialgebra aspect begins with a $K$-algebra $H$, 
an algebra over the field $K$, having the basic operations and maps such as 
multiplication, addition, and a unit from the underlying field.  In addition
to the regular $K$-algebra structure, $H$ has a coalgebra structure as well.  The 
coalgebra structure requires that $H$ have two more maps associated with it:
the comultiplication map which is a map from $H$ to $H \otimes H$($H$ tensor $H$) and the
counit map which maps $H$ to $K$.  These maps and the 
$K$-algebra maps obey certain rules to form the bi-algebra structure.

The existence of the antipode depends upon the bialgebra structure.  When the
antipode exists it is a map from $H$ to $H$, and the map's action is uniquely derived from
a formula involving the maps of the bialgebra.
For more information about coalgebras, 
bialgebras, Hopf algebras and antipodes consult:

Montgomery, Susan. Hopf Algebras and their Actions on Rings.
Regional Conference Series in Mathematics Number 82 (1992) 1-10

The antipode acts as an antihomomorphism from $H$ to $H$.  To make inputting Hopf algebras easier, we have provided functions for working with algebra antihomomorphisms.

%%%%%%%%%%%%%%%%%%%%%%%%%%%%%%%%%%%%%%%%%%%%%%%%%%%%%
\Section{Antihomomorphisms}

An antihomomorphism $S$ from the $K$-algebra $A$ to the $K$-algebra $B$ is a 
$K$-linear map from $A$ to $B$.  Further, $S$ has the property that for all $a$ in $A$ and $b$ in $B$,
$S(a*b)=S(b)*S(a)$.  Thus, $S$ reverses multiplication instead of preserving multiplication as a 
homomorphism would.  Another way of thinking about an antihomomorphism is that it is a 
homomorphism from $A$ to the opposite algebra of $B$ where multiplication is reversed.

\>AlgebraAntihomomorphismByImages(<A>, <B>, <Generators>, <Images>) F
\>AlgebraAntihomomorphismByImagesNC(<A>, <B>, <Generators>, <Images>) F

This function creates the antihomomorphism from $A$ to $B$
where a generator of $A$ in the list of generators is mapped to the image in $B$ at the same
position in the list of images.  <Generators> must be a subset of $A$ and <Images> must be a subset of $B$.
If the given generators and images would not specify an antihomomorphism, this function will instead return fail.
Infinite dimensional algebras should not be used with this function; `AlgebraAntihomomorphismByImagesNC' should be used.
  It is a nonchecking version of `AlgebraAntihomomorphismByImages'.

\>AlgebraWithOneAntihomomorphismByImages(<A>, <B>, <Generators>, <Images>) F
\>AlgebraWithOneAntihomomorphismByImagesNC(<A>, <B>, <Generators>, <Images>) F

These two functions create antihomomorphisms as above but can potentially use the information that the given algebra is an algebra-with-one.
The $1$ of $A$ will always be mapped to the $1$ of $B$.

\Section{Categories of Antihomomorphisms}

\>IsAlgebraAntihomomorphism(<f>) C

This category contains all antihomomorphisms.

\>IsAlgebraWithOneAntihomomorphism(<f>) C

This category contains all antihomomorphisms between algebras-with-one.

%%%%%%%%%%%%%%%%%%%%%%%%%%%%%%%%%%%%%%%%%%%%%%%%%%%%
\Section{Group Rings as Hopf Algebras}

A group ring, $KG$, where $K$ is a field and $G$ is a group, has a natural
Hopf algebra structure.  Taking the natural addition, multiplication
and unit in the algebra, the coalgebra structure is as follows.
Comultiplication from $KG$ to $KG \otimes KG$ maps $g$ to $g \otimes g$
for all $g$ in $G$.  The counit map from $KG$ to $K$ takes $0$ to $0$
and all non-zero elements to $1$.  The antipode of the Hopf algebra
$KG$ is the antihomomorphism from $KG$ to $KG$ taking $g$ to $g^{-1}$.
Since these are basic Hopf algebras where the structure is already known,
the following two functions will automatically create the Hopf algebra
structure from a group ring.

\>`HopfAlgebra(<KG>)'{Group Algebras as Hopf Algebras}  F
\>`HopfAlgebraNC(<KG>)' F

The following is an example of such a (natural) construction; a few
functions, later described, are used here to deepen the example.

\beginexample
gap> G := SymmetricGroup(4);
Sym( [ 1 .. 3 ] )
gap> A := GroupRing
    GroupRing
gap> A := GroupRing(Rationals,G);
<algebra-with-one over Rationals, with 2 generators>
gap> IsGroupRing(A);
true
gap> H := HopfAlgebra(A);
<Hopf algebra from <algebra-with-one over Rationals, with 2 generators>>
gap> IsHopfAlgebra(H);
true
gap> UnderlyingAlgebra(H);
<algebra-with-one over Rationals, with 2 generators>
gap> ComultiplicationMap(H);     
[ (1)*(), (1)*(2,3), (1)*(1,2), (1)*(1,2,3), (1)*(1,3,2), (1)*(1,3) ] -> 
[ 1*((1)*()<x>(1)*()), 1*((1)*(2,3)<x>(1)*(2,3)), 1*((1)*(1,2)<x>(1)*(1,2)), 
  1*((1)*(1,2,3)<x>(1)*(1,2,3)), 1*((1)*(1,3,2)<x>(1)*(1,3,2)), 1*((1)*(1,3)<x>(1)*(1,3)) ]
gap> CounitMap(H);
[ (1)*(), (1)*(2,3), (1)*(1,2), (1)*(1,2,3), (1)*(1,3,2), (1)*(1,3) ] -> [ 1, 1, 1, 1, 1, 1 ]
gap> Aimgs := AntipodeMap(H);
[ (1)*(), (1)*(2,3), (1)*(1,2), (1)*(1,2,3), (1)*(1,3,2), (1)*(1,3) ] -> 
[ (1)*(), (1)*(2,3), (1)*(1,2), (1)*(1,3,2), (1)*(1,2,3), (1)*(1,3) ]
gap> B := GeneratorsOfAlgebra(H);
[ (1)*(), (1)*(1,2,3), (1)*(1,2) ]
gap> B[2]^2 in H;    
true
gap> B[2]^2;
(1)*(1,3,2)
gap> B[2]^2*B[3];
(1)*(1,3)
gap> Comultiply(B[2]^2+B[3]);
1*((1)*(1,2)<x>(1)*(1,2))+1*((1)*(1,3,2)<x>(1)*(1,3,2))
gap> Counit(B[2]^2+B[3]);    
2
gap> Antipode(B[2]^2+B[3]);
(1)*(1,2)+(1)*(1,2,3)

\endexample

The checking version ensures that the group ring is over a field.
The non-checking version does not verify that the group ring will form a Hopf algebra.
These functions will return the Hopf algebra with maps as described above.

%%%%%%%%%%%%%%%%%%%%%%%%%%%%%%%%%%%%%%%%%%%%%%%%%%%%%%%
\Section{Creating Hopf Algebras}

If you want to create Hopf algebras other than group rings you must use the following 
function.  

\>HopfAlgebraNC(<A>,<CoMult>,<Counit>,<Anti>) F

This function creates the Hopf algebra using the 
algebra <A>, the comultiplication map $CoMult$, the counit map $Counit$ and the 
antipode map $Anti$.  You must supply the maps and they are not checked to ensure that they form a Hopf algebra. 
We intend to add a checking version of this function in the future.

The Hopf algebra returned will be a new algebra unassociated with
the original algebra $A$.
Elements taken from $A$ cannot be used with the corresponding Hopf algebra.

Below is an example of the creating the Hopf algebra where the base algebra is $Q(KG)$.

\beginexample
gap> q:=Quiver(1,[[1,1]]);
<quiver with 1 vertices and 1 arrows>
gap> p:=PathAlgebra(Rationals,q);
<algebra-with-one over Rationals, with 2 generators>
gap> b:=GeneratorsOfAlgebra(p);
[ (1)*v1, (1)*a1 ]
gap> v:=b[1]; 
(1)*v1
gap> x:=b[2];
(1)*a1
gap> pt:=TensorProductOfAlgebras(p,p);
<tensor product of 2 algebras, with 3 generators>
gap> deltx:=TensorElement(pt,[v,x])+TensorElement(pt,[x,v]); 
1*((1)*v1<x>(1)*a1)+1*((1)*a1<x>(1)*v1)
gap> delta:=AlgebraHomomorphismByImagesNC(p,pt,b,[One(pt),deltx]);
[ (1)*v1, (1)*a1 ] -> 
[ 1*((1)*v1<x>(1)*v1), 1*((1)*v1<x>(1)*a1)+1*((1)*a1<x>(1)*v1) ]
gap> counit:=AlgebraHomomorphismByImagesNC(p,Rationals,b,[One(Rationals),Zero(Rationals)]);
[ (1)*v1, (1)*a1 ] -> [ 1, 0 ]
gap> s:=AlgebraAntihomomorphismByImagesNC(p,p,b,[v,-x]);
MappingByFunction( <algebra-with-one over Rationals, with 
2 generators>, <algebra-with-one over Rationals, with 
2 generators>, function( e ) ... end )
gap> H:=HopfAlgebraNC(p,delta,counit,s);
<Hopf algebra from <algebra-with-one over Rationals, with 2 generators>>
\endexample

%%%%%%%%%%%%%%%%%%%%%%%%%%%%%%%%%%%%%%%%%%%%%%%%%%%%%%%%
\Section{Categories and Attributes of Hopf Algebras}

\>IsHopfAlgebra(<H>) C

This category contains all Hopf algebras.

\>IsHopfAlgebraElement(<h>) C

This category contains all Hopf algebra elements.

\>IsHopfAlgebraBasis(<b>) C

This category contains all finite-dimensional Hopf algebra bases.

\>UnderlyingAlgebra(<H>) A

Returns the underlying algebra of the Hopf algebra $H$.

\>ComultiplicationMap(<H>) A

Returns the comultiplication map of the Hopf algebra.

\>CounitMap(<H>) A

Returns the counit map of the Hopf algebra.

\>AntipodeMap(<H>) A

Returns the antipode map of the Hopf algebra.

%%%%%%%%%%%%%%%%%%%%%%%%%%%%%%%%%%%%%%%%%%%%%%%%%%%%%%%%%
\Section{Hopf Algebra Operations}

In addition to the normal algebraic element operations such as addition, 
subtraction, and multiplication there are three
important operations specific to Hopf algebra elements.

\>Comultiply(<h>) O

Comultiply will apply the comultiplication map of the algebra containing $h$ to $h$.

\>Counit(<h>) O

Counit will apply the counit map of the algebra containing $h$ to $h$.

\>Antipode(<h>) O

Antipode will apply the antipode map of the algebra containing $h$ to $h$.

The standard operations on algebras such as Dimension, IsFiniteDimensional, Basis, Coefficients, 
and in are defined for Hopf algebras and delegate to the corresponding functions in the underlying algebra.  
Refer to chapter "ref:algebras" for more information.

Continuing the example above shows some applications of the operations.

\beginexample
gap> hb:=GeneratorsOfAlgebra(H);
[ (1)*v1, (1)*v1, (1)*a1 ]
gap> Comultiply(hb[3]);
1*((1)*v1<x>(1)*a1)+1*((1)*a1<x>(1)*v1)
gap> Comultiply(hb[3]+hb[3]*hb[3]);
1*((1)*v1<x>(1)*a1)+1*((1)*v1<x>(1)*a1^2)+1*((1)*a1<x>(1)*v1)+2*((1)*a1<x>(
1)*a1)+1*((1)*a1^2<x>(1)*v1)
gap> Counit(hb[3]);
0
gap> Counit(hb[2]);
1
gap> Antipode(hb[3]^3);
(-1)*a1^3
\endexample
