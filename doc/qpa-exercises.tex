\documentclass[a4paper]{amsart}
\usepackage[all]{xy}
\usepackage{exercise}

\newenvironment{theoback}
{\medskip\footnotesize\textit{Theoretic background.} }
{\qed\par\medskip}
\newenvironment{theocomm}
{\medskip\footnotesize\textit{Theoretic comment.} }
{\qed\par\medskip}

\newcommand{\VV}[2]{\begin{pmatrix} #1 & #2 \end{pmatrix}}
\newcommand{\vv}[2]{\left( \begin{smallmatrix} #1 & #2 \end{smallmatrix} \right)}

\newcommand{\Z}{\mathbb{Z}}

\begin{document}

\title{QPA exercises}

\date{\today}

\maketitle

Many of the exercises can be solved both using QPA and by hand.  It
can be useful to solve them in both ways and check that the results
agree.

The variable $k$ stands for an arbitrary field.  Whenever the field is
called $k$, it does not matter which field you use.  You can for
example use the field of rational numbers, which is called
\texttt{Rationals} in GAP.

\begin{Exercise}[title={Paths and algebra elements}]
Let $Q$ be the following quiver:
\[
\xymatrix@R=1ex{
&& 3 \\
1 \ar@<1ex>[r]^a \ar@<-1ex>[r]_b &
2 \ar[ur]^c \ar[dr]_d \\
&& 4
}
\]
\Question How many paths of length~$0$ does $Q$ have?  How many of
length~$1$, $2$ and~$3$?
\Question Construct the quiver~$Q$ and the path algebra $kQ$ in QPA.
\Question What is the $k$-dimension of the path algebra $kQ$?
\Question Simplify the following expression in $kQ$:
\[
(v_1 - v_2 + a)(ac + b + d) + (bc - a)(b + c)d
\]
\end{Exercise}

\begin{Exercise}[title={Admissible ideals}]
\Question
For each quiver $Q_i$ and ideal $I_i \subseteq kQ_i$ given below,
determine if the ideal is admissible.
\begin{flalign*}
Q_1&\colon
\vcenter{\xymatrix@R=2ex@C=1ex{
& 2 \ar[dr]^b \\
1 \ar[ur]^a &&
3 \ar[ll]^c
}}
&
I_1 &= \langle bc \rangle
&
Q_2&\colon
\vcenter{\xymatrix@C=3ex{
1 \ar@<1ex>[r]^a &
2 \ar@<1ex>[r]^b \ar@<1ex>[l]^d &
3 \ar@<1ex>[l]^c
}}
&
I_2 &= \langle bc \rangle
\\
Q_3&\colon
\vcenter{\xymatrix@R=2ex@C=2ex{
1 \ar[r]^a & 2 \ar[d]^b \\
4 & 3 \ar[l]^c
}}
&
I_3 &= \langle abc \rangle
&
Q_4&\colon
\vcenter{\xymatrix@R=2ex@C=1ex{
& 2 \ar[dr]^b \\
1 \ar[ur]^a \ar[rr]_c &&
3 \ar[rr]_d &&
4
}}
&
I_4 &= \langle cd, ab - c \rangle
\end{flalign*}
\[
Q_5\colon
\vcenter{\xymatrix@R=2ex@C=1ex{
& 1 \ar[dl]_{a_2} \\
2 \ar[rr]_c &&
3 \ar[ul]_{a_1} \ar[d]^{b_1} \\
4 \ar[u]^{b_3} &&
5 \ar[ll]^{b_2}
}}
\qquad
I_5 = \langle a_1 a_2 - b_1 b_2 b_3, b_3 c a_1 \rangle
\]
% \[
% Q_1\colon
% \vcenter{\xymatrix@R=2ex@C=1ex{
% & 2 \ar[dr]^b \\
% 1 \ar[ur]^a &&
% 3 \ar[ll]^c
% }}
% \qquad
% I_1 = \langle bc \rangle
% \]
% \[
% Q_1\colon
% \vcenter{\xymatrix@C=3ex{
% 1 \ar@<1ex>[r]^a &
% 2 \ar@<1ex>[r]^b \ar@<1ex>[l]^d &
% 3 \ar@<1ex>[l]^c
% }}
% \qquad
% I_1 = \langle bc \rangle
% \]
% \[
% Q_1\colon
% \vcenter{\xymatrix@C=2ex{
% 1 \ar[r]^a & 2 \ar[r]^b & 3 \ar[r]^c & 4
% }}
% \qquad
% I_1 = \langle abc \rangle
% \]
% \[
% Q_1\colon
% \vcenter{\xymatrix@R=2ex@C=1ex{
% & 2 \ar[dr]^b \\
% 1 \ar[ur]^a \ar[rr]_c &&
% 3 \ar[rr]_d &&
% 4
% }}
% \qquad
% I_1 = \langle cd, ab - c \rangle
% \]
% \[
% Q_2\colon
% \vcenter{\xymatrix@R=2ex@C=1ex{
% & 1 \ar[dl]_{a_2} \\
% 2 \ar[rr]_c &&
% 3 \ar[ul]_{a_1} \ar[d]^{b_1} \\
% 4 \ar[u]^{b_3} &&
% 5 \ar[ll]^{b_2}
% }}
% \qquad
% I_2 = \langle a_1 a_2 - b_1 b_2 b_3, b_3 c a_1 \rangle
% \]

\Question
Show that the zero ideal in a path algebra $kQ$ is admissible if and
only if the quiver $Q$ does not contain any oriented cycle.
\end{Exercise}

\begin{Exercise}[title={Modules}]
Let $Q$ be the following quiver:
\[
\xymatrix{
1 \ar[r]^a &
2 \ar@<1ex>[r]^b \ar@<-1ex>[r]_c &
3
}
\]
\Question Construct the path algebra $kQ$ in QPA.
\Question Construct the following $kQ$-modules in QPA:
\[
X =
\xymatrix{
k \ar[r]^1 &
k \ar@<1ex>[r]^{\vv{1}{0}}
  \ar@<-1ex>[r]_{\vv{0}{1}} &
k^2
}
\qquad
Y =
\xymatrix{
k \ar[r]^{\vv{1}{0}} &
k^2 \ar@<1ex>[r]^{\left(\begin{smallmatrix} 1&0 \\ 0&1 \end{smallmatrix}\right)}
    \ar@<-1ex>[r]_{\left(\begin{smallmatrix} 1&0 \\ 0&0 \end{smallmatrix}\right)} &
k^2
}
\qquad
Z =
\xymatrix@C=1.5em{
k \ar[r]^1 &
k \ar@<1ex>[r]^{1}
  \ar@<-1ex>[r]_{1} &
k
}
\]
\Question Let $y$ be the following element of the module $Y$:
\[
\xymatrix{
4 \ar[r] &
{\VV{2}{3}} \ar@<1ex>[r] \ar@<-1ex>[r] &
{\VV{7}{1}}
}
\]
Compute the action of the algebra element $(3ab + 4c - 2v_3)$ on $y$.
\Question Are any of the modules $X$, $Y$ and $Z$ projective?
\Question Let $A = kQ/\langle ab - ac \rangle$.  Construct the algebra
$A$ in QPA.
\Question For each of the modules $X$, $Y$ and $Z$: Does the module
respect the relation $ab - ac$?  If so, construct the corresponding
$A$-module in QPA.
\Question Are any of the $A$-modules you just constructed projective?
\end{Exercise}

% TODO: making homomorphisms -- checking if something is a homomorphism

% TODO: Nakayama/truncated algebra?

% TODO: submodules

% TODO: radical, top, socle

\begin{Exercise}[title={Isomorphisms}]
Let $k$ be the field of rational numbers and let $A =
kQ/\langle\rho\rangle$ be the algebra given by the following quiver
and relations:
\[
Q\colon
\vcenter{\xymatrix@C=1em@R=1em{
1 \ar[rr]^a &&
2 \ar[dl]^b \\
& 3 \ar[ul]^c
}}
\qquad
\rho = \{ abc \}
\]
Consider the following $kQ$-modules.
\begin{align*}
M_1&\colon
\vcenter{\xymatrix@C=1em@R=1em{
k^2 \ar[rr]^{\left( \begin{smallmatrix} 0 & 1 \\ 3 & 5 \end{smallmatrix} \right)} &&
k^2 \ar[dl]^{\left( \begin{smallmatrix} 0 & 0 \\ 1 & 0 \end{smallmatrix} \right)} \\
&
k^2 \ar[ul]^{\left( \begin{smallmatrix} 0 & 0 \\ 2 & 0 \end{smallmatrix} \right)}
}}
&
M_2&\colon
\vcenter{\xymatrix@C=1em@R=1em{
k^2 \ar[rr]^{\left( \begin{smallmatrix} 0 & 0 \\ 2 & 0 \end{smallmatrix} \right)} &&
k^2 \ar[dl]^{\left( \begin{smallmatrix} 0 & 1 \\ 0 & 0 \end{smallmatrix} \right)} \\
&
k^2 \ar[ul]^{\left( \begin{smallmatrix} 0 & 1 \\ 0 & 0 \end{smallmatrix} \right)}
}}
\\
M_3&\colon
\vcenter{\xymatrix@C=1em@R=1em{
k^2 \ar[rr]^{\left( \begin{smallmatrix} -1 &  0 \\ 1 & 3 \end{smallmatrix} \right)} &&
k^2 \ar[dl]^{\left( \begin{smallmatrix} -1 & -1 \\ 2 & 2 \end{smallmatrix} \right)} \\
&
k^2 \ar[ul]^{\left( \begin{smallmatrix} 2 & 0 \\ -2 & 0 \end{smallmatrix} \right)}
}}
&
M_4&\colon
\vcenter{\xymatrix@C=1em@R=1em{
k^2 \ar[rr]^{\left( \begin{smallmatrix} -2 & 0 \\ 2 & 0 \end{smallmatrix} \right)} &&
k^2 \ar[dl]^{\left( \begin{smallmatrix} 1 & 0 \\ 0 & 0 \end{smallmatrix} \right)} \\
&
k^2 \ar[ul]^{\left( \begin{smallmatrix} 0 & 0 \\ 0 & 1 \end{smallmatrix} \right)}
}}
\end{align*}
\Question Are any of these modules isomorphic to each other?
\Question For the modules that are isomorphic, find an isomorphism.
\end{Exercise}


% ?
% given some modules, which module is the direct sum of ...;
% find inclusions/projections.
% (need to use IsomorphismOfModules and DirectSumInclusions)

% TODO: simple, projective modules

% TODO: kernel, cokernel, image

\begin{Exercise}[title={Chain complexes}]
\Question Let $A = kQ/I$ where $Q$ and $I$ are as follows:
\[
Q\colon
\vcenter{\xymatrix{
1 \ar[d]^a \\
2 \ar@(dl,dr)^b
}}
\qquad
I = \langle ab, b^2 \rangle
\]
Let $C$ be the complex of $A$-modules where every object $C_i$ is the
same module
\[
C_i =
\vcenter{\xymatrix{
k^2 \ar[d]^{\left( \begin{smallmatrix} 0&1 \\ 0&0 \end{smallmatrix} \right)} \\
k^2 \ar@(dl,dr)_{\left( \begin{smallmatrix} 0&1 \\ 0&0 \end{smallmatrix} \right)}
}}
\]
and the differential $d_i$ is given by
\[
d_i \colon
\vcenter{\xymatrix@C=4em{
k^2 \ar[d]^{\left( \begin{smallmatrix} 0&1 \\ 0&0 \end{smallmatrix} \right)}
    \ar[r]^{\left( \begin{smallmatrix} 0&i \\ 0&0 \end{smallmatrix} \right)} &
k^2 \ar[d]^{\left( \begin{smallmatrix} 0&1 \\ 0&0 \end{smallmatrix} \right)} \\
k^2 \ar@(dl,dr)_{\left( \begin{smallmatrix} 0&1 \\ 0&0 \end{smallmatrix} \right)}
    \ar[r]_{\left( \begin{smallmatrix} 0&i \\ 0&0 \end{smallmatrix} \right)} &
k^2 \ar@(dl,dr)_{\left( \begin{smallmatrix} 0&1 \\ 0&0 \end{smallmatrix} \right)}
}}
\]
for every $i \in \Z$.

\Question Repeat the above, but replace both matrices
$\left( \begin{smallmatrix} 0&i \\ 0&0 \end{smallmatrix} \right)$
in $d_i$ by
$\left( \begin{smallmatrix} 0&1 \\ 0&0 \end{smallmatrix} \right)$.
\end{Exercise}

% TODO: projective modules, resolutions

% AR-theory
\begin{Exercise}[title={Almost split sequences}]
Let $Q=\xymatrix{ 1\ar@(ul,dl)_a \ar@<1ex>[r]^b &
  2\ar@<1ex>[l]^c }$, and let 
\[\Lambda = \mathbb{Z}_3Q/J^2,\] 
where $J$ is the ideal in $\mathbb{Z}_3Q$ generated by the
arrows. 

\Question Compute the almost split sequence ending in the simple
module associated to vertex $1$.

\Question Compute the almost splist sequence ending in the simple
module associated to vertex $2$. 

\Question Check that the middle term of the two above almost split
sequences are injective modules, and that they have a common direct
summand. 

\Question Challenge: Show that the Auslander-Reiten quiver of $\Lambda$ is 
\[\xymatrix{
 & \bullet\ar@{..}[rr]\ar[dr] &         & \bullet\ar[dr] & \\
\bullet \ar@{..}[rr]\ar[ur]\ar[dr] & & \bullet \ar@{..}[rr]\ar[ur]\ar[dr] && \circ \\
 & \bullet\ar@{..}[rr]\ar[ur]\ar[dr] &         & \bullet \ar[ur]\ar[dr] & \\
\bullet\ar@{..}[rr]\ar[ur] & & \bullet \ar@{..}[rr]\ar[ur] && \circ 
}\]
where the extreme vertices $\bullet$ and $\circ$ are identified. 
\end{Exercise}

\end{document}
